\documentclass[12pt]{article}
\usepackage[letterpaper, top=1.25in, bottom=1.25in, left=1in, right=1in]{geometry}
\usepackage{fancyhdr} % http://ctan.org/pkg/fancyhdr
\usepackage{hyperref}
% \usepackage{xcolor}
\usepackage{graphicx}
\usepackage{indentfirst}

% % % % % % % % % % % % 
% Citation formatting %
% % % % % % % % % % % % 

\usepackage[style=ieee]{biblatex}
\addbibresource{sources.bib}

% % % % % % % % %
% Hyperlinking  %
% % % % % % % % %

\hypersetup{
    colorlinks=false, % set true if you want colored links
    linktoc=all,     % set to all if you want both sections and subsections linked
    % linkcolor=black,  %c hoose some color if you want links to stand out
    % citecolor=black
    % urlcolor=black
}

% % % % % % % % % % % % % % % %
% Global title, author, date  %
% % % % % % % % % % % % % % % %

\title{Exploring Smart Sensor Systems}
\author{Nathan Quadras \& Kosta Sergakis}
% \author1{Quadras \& Sergakis}
% \date{\today}
\date{April 30, 2025}
\makeatletter
\let\runauthor\@author
\let\runtitle\@title
\let\rundate\@date
\makeatother

% % % % % % % % % % %
% Header and Footer %
% % % % % % % % % % %

\pagestyle{fancy} % change page style to fancy
\fancyhf{} % clear header/footer
\setlength{\headheight}{15pt}
\fancyhead[L]{Quadras \& Sergakis}
\fancyhead[C]{\runtitle}
\fancyhead[R]{\rundate}
\fancyfoot[C]{\thepage} % \fancyfoot[R]{\thepage}
\renewcommand{\headrulewidth}{0.4pt} % default \headrulewidth is 0.4pt
\renewcommand{\footrulewidth}{0.4pt} % default \footrulewidth is 0pt

% \usepackage{natbib}
\begin{document}

    % % % % % % % %
    % Title Page  %
    % % % % % % % %
    
    \begin{titlepage}
        \begin{center}
            \vspace*{1.5cm}
            
            \textbf{\runtitle}
            
            \vspace{1cm}
            
            An exploration of:\\
            \vspace{0.25cm}
            Slip Angle Sensors
                
            \vspace{1cm}
            
            \textbf{\runauthor}
            
            \vfill  
            
            \vspace{1cm}

            \includegraphics[width=0.4\textwidth]{resources/michigan-state-logo-png-transparent.png}
                
            Electrical and Computer Engineering\\
            Michigan State University\\
            East Lansing, Michigan\\
            \rundate
                
        \end{center}
    
    \end{titlepage}


    % % % % % % % % % % %
    % Table of Contents %
    % % % % % % % % % % %

    \setcounter{secnumdepth}{0} % removes section numbers
    % \maketitle
    
    \tableofcontents
    
    \newpage

    \section{Overview \& Motivation}
        
        Vehicles are more attainable than ever and are advancing quickly. Modern cars are equipped with 
        sophisticated systems that enhance performance, safety, and driver comfort. Vehicle dynamics is a tremendous 
        topic and an area of growing interest, especially as more automated driving technologies emerge. Vehicle 
        dynamics is the study of vehicle motion in relevant user operations [1], which encompasses factors like 
        kinematics, forces, and moments acting on a vehicle during acceleration, braking and steering. The core of vehicle 
        dynamics involves the following primary aspects: the mechanisms that disturb a vehicle’s state (inputs) and the 
        mechanisms through which the vehicle responds (outputs). Arguably the most critical aspect of vehicle dynamics is 
        tire behavior, specifically how a tire generates lateral force during cornering. Central to this behavior is the 
        concept of slip angle – the difference between the direction a vehicle is traveling and the direction that the body 
        of the vehicle is pointing (heading vs. true heading) [2]. Understanding slip angle is essential for analyzing 
        handling characteristics, improving stability control systems, and optimizing driver feedback.
        
        \subsection{Vehicle Dynamics}


    \autocite{social-media-family}

    % % % % % % % % %
    % Bibliography  %
    % % % % % % % % %

    % \addcontentsline{toc}{section}{Works Cited}
    % \printbibliography
    \newpage
    \phantomsection % Create a phantom section to get the correct page number in the ToC
    \addcontentsline{toc}{section}{Works Cited}
    \printbibliography[title=Works Cited]

    % \bibliographystyle{plainnat}
    % \bibliography{sources}

\end{document}